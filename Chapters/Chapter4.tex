\chapter{Ensayos y resultados}

\label{cap:EnsayosResultados}

En este capítulo se describe la fase de investigación previa al desarrollo del dispositivo de captura, junto con las pruebas realizadas para validar su correcto funcionamiento, tanto en un entorno de laboratorio como en una formación de Trenes Argentinos.

\section{Capturas del tráfico de la red en una formación}
\label{sec:capturas}

Como parte de la etapa de investigación, se realizó en junio de 2020 una visita al taller Victoria de Trenes Argentinos, coordinada por Sergio Dieleke (Coordinador Laboratorio Electrónico, Subgerencia de Material Rodante Línea Mitre). El objetivo principal de la visita fue tomar capturas del tráfico del bus MVB, para tomar conocimiento acerca del estándar TCN y de la implementación particular en las EMU de Trenes Argentinos.

Para tomar las capturas se conectó un MAX485 y un analizador lógico VKTECH entre dos dispositivos MVB.
También se utilizó un osciloscopio para verificar que la señal capturada tuviera las características esperadas.
En la figura~\ref{fig:banco-capturas} se muestra un diagrama de bloques del banco de medición.
En las figuras~\ref{fig:foto-banco-capturas} y \ref{fig:osciloscopio} se muestra una fotografía del banco de medición y un detalle de la señal capturada en el osciloscopio.

% video con la secuencia https://drive.google.com/drive/folders/1I-V33ElLX13Iy0YliUeojRwYFQKuucAO
\begin{figure}[htbp]
	\centering
    {
        \fontfamily{phv}
        \fontsize{9pt}{9pt}\selectfont
        \input{./Figures/banco-captura.pdf_tex}
    }
	\caption{Banco de medición utilizado para tomar las capturas.}
    \label{fig:banco-capturas}
\end{figure}

\begin{figure}[htbp]
	\centering
	\includegraphics[width=0.95\textwidth]{./Figures/foto-capturas.jpg}
	\caption{Fotografía del banco de medición utilizado para tomar las capturas.}
    \label{fig:foto-banco-capturas}
\end{figure}

\begin{figure}[htbp]
	\centering
	\includegraphics[width=0.95\textwidth]{./Figures/osciloscopio.jpg}
	\caption{Señal MVB capturada en el osciloscopio.}
    \label{fig:osciloscopio}
\end{figure}

\section{Decodificación de las capturas}
\label{sec:decodificacion}

\section{Pruebas con el generador de señal}
\section{Captura en tiempo real}
