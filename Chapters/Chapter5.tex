\chapter{Conclusiones}

\label{cap:Conclusiones}

En este capítulo se hace un breve resumen de los resultados obtenidos a partir de la realización de este trabajo, y luego se identifican algunas oportunidades de mejora para un trabajo futuro.

\section{Resultados obtenidos}

El dispositivo de captura desarrollado cumple con todos los requerimientos planteados en la sección~\ref{sec:requerimientos}, ya que permite capturar el tráfico del bus MVB para visualizar en tiempo real o para almacenar la evolución histórica.
El dispositivo fue probado exitosamente en una formación ferroviaria detenida en un taller de Trenes Argentinos.

Este trabajo cubre una de las necesidades de Trenes Argentinos, que era la de almacenar el tráfico del bus.
Además está en línea con los objetivos del GICSAFe: evitar la dependencia de tecnología cerrada, sustituir importaciones por desarrollos propios y generar trabajo con alto valor agregado.

Más allá del desarrollo del dispositivo de captura, la importancia de este trabajo radica principalmente en el conocimiento adquirido acerca del funcionamiento de la red TCN en las formaciones de Trenes Argentinos.
Esto abre el camino para desarrollar en el futuro dispositivos con capacidades más avanzadas.
En la siguiente sección se enumeran algunas posibilidades.

\section{Trabajo futuro}

A continuación se enumeran algunas oportunidades de mejora para un posible trabajo futuro:

\begin{itemize}
    \item El dispositivo fue probado en una formación ferroviaria detenida. Queda pendiente hacer pruebas con una formación en movimiento.

    \item Si bien el dispositivo de captura cuenta con una interfaz Wi-Fi, por el momento no ofrece la posibilidad de monitorear el tráfico del bus MVB a distancia, por ejemplo desde un taller de Trenes Argentinos. Esto se podría lograr agregando una interfaz para transmitir los datos mediante la red de datos móviles (por ejemplo 4G).

    Por otro lado, el estándar Wi-Fi puede no ser el más óptimo para un entorno como el de una formación ferroviaria, en el que el ruido eléctrico puede afectar la transmisión inalámbrica. Se necesita más investigación y pruebas con respecto a esto.

    \item El dispositivo está diseñado para capturar el tráfico del bus MVB implementado con capa física EMD. Actualmente no es posible capturar el tráfico en segmentos WTB, o MVB con otras capas físicas (por ejemplo ESD).

    \item El software de captura analiza únicamente las tramas de tipo \texttt{Process\_Data}, que son las que publican el valor de las distintas variables. En las capturas realizadas se observó que este tipo de tramas son las que se transmiten con mayor frecuencia (ver figura~\ref{fig:interactivo}). En caso de que los otros tipos de trama sean relevantes en el futuro, habrá que agregar al software la capacidad de analizarlos.

    \item Como se explicó en la sección~\ref{sec:propuesta}, por simplicidad se decidió desarrollar el dispositivo de captura basándose en un microcontrolador o procesador en lugar de un componente especializado como el chip MVBC02C.
    Esta solución tiene la desventaja de cargar al procesador con procesamiento que se podría hacer por hardware en lugar de software. Esto podría ser problemático si se quisiera agregar al dispositivo procesamientos adicionales por software.
    Una alternativa podría ser mover la decodificación del protocolo MVB a un dispositivo programable como una FPGA.
\end{itemize}

